\documentclass[12pt,t]{beamer}
\usepackage{graphicx}
\usepackage{bm}
\usepackage{xcolor}
\usepackage{multirow}
\setbeameroption{hide notes}
\setbeamertemplate{note page}[plain]

% get rid of junk
\usetheme{default}
\beamertemplatenavigationsymbolsempty
\hypersetup{pdfpagemode=UseNone} % don't show bookmarks on initial view

% tables
\usepackage{tabularx}
\newcolumntype{Y}{>{\centering\arraybackslash}X}

% font
\usefonttheme{professionalfonts}
\usefonttheme{serif}
\usepackage{fontspec}
\setmainfont{Helvetica Neue}
\setbeamerfont{note page}{family*=pplx,size=\footnotesize} % Palatino for notes

% named colors
\definecolor{offwhite}{RGB}{249,242,215}
\definecolor{foreground}{RGB}{25,25,25}
\definecolor{background}{RGB}{250,250,245}
\definecolor{title}{RGB}{20,20,20}
\definecolor{gray}{RGB}{80,80,80}
\definecolor{subtitle}{RGB}{20,20,20}
\definecolor{hilight}{RGB}{255,127,0}
\definecolor{vhilight}{RGB}{255,111,207}
\definecolor{lolight}{RGB}{155,155,155}
%\definecolor{green}{RGB}{125,250,125}

% use those colors
\setbeamercolor{titlelike}{fg=title}
\setbeamercolor{subtitle}{fg=subtitle}
\setbeamercolor{institute}{fg=gray}
\setbeamercolor{normal text}{fg=foreground,bg=background}
\setbeamercolor{item}{fg=foreground} % color of bullets
\setbeamercolor{subitem}{fg=gray}
\setbeamercolor{itemize/enumerate subbody}{fg=gray}
\setbeamertemplate{itemize subitem}{{\textendash}}
\setbeamerfont{itemize/enumerate subbody}{size=\footnotesize}
\setbeamerfont{itemize/enumerate subitem}{size=\footnotesize}

% page number
\setbeamertemplate{footline}{%
    \raisebox{5pt}{\makebox[\paperwidth]{\hfill\makebox[20pt]{\color{gray}
          \scriptsize\insertframenumber}}}\hspace*{5pt}}

% add a bit of space at the top of the notes page
\addtobeamertemplate{note page}{\setlength{\parskip}{12pt}}

% a few macros
\newcommand{\bi}{\begin{itemize}}
\newcommand{\ei}{\end{itemize}}
\newcommand{\ig}{\includegraphics}
\newcommand{\subt}[1]{{\footnotesize \color{subtitle} {#1}}}

% title info
\title{Local variable selection and parameter estimation for spatially varying coefficient models}
%\subtitle{A researcher's perspective}
\author{\href{http://www.somesquares.org}{Wesley Brooks}}
\institute{\href{http://www.stat.wisc.edu}{Department of Statistics} \\[2pt] \href{http://www.wisc.edu}{University of Wisconsin{\textendash}Madison}}
\date{\href{http://www.somesquares.org}{\tt \scriptsize somesquares.org}}


%bibliography stuff
\usepackage[autostyle]{csquotes}
\usepackage[backend=biber,style=authoryear,natbib=true]{biblatex}
\addbibresource{../../references/gwr.bib}


\AtBeginSection{\frame{\sectionpage}}



\begin{document}

% title slide
{
  \setbeamertemplate{footline}{} % no page number here
  \frame{
    \titlepage
    \note{These slides were prepared for a practice version of my preliminary exam to advance to Ph.D candidacy in statistics at the University of Wisconsin{\textendash}Madison.}
  }
}



\section{Motivation}



\begin{frame}{Motivation}
\subt{Take a look at some data}

\begin{center}
  \ig[width=0.6\textwidth]{../../figures/practice-talk/poverty-response}
\end{center}

\note{This is the county-level poverty rate from 1970, as well as the proportion of people who worked in manufacturing, agriculture, and services.

How is this data to be analyzed?}
\end{frame}




\begin{frame}{Motivation}
\subt{Take a look at some data}

\bigskip
\begin{center}
  \ig[width=\textwidth]{../../figures/practice-talk/poverty-covariates}
\end{center}

\note{
This is the county-level poverty rate from 1970, as well as the proportion of people who worked in manufacturing, agriculture, and services.

How is this data to be analyzed?
}
\end{frame}





\begin{frame}{Motivation}
\subt{Sensible questions about the data}

\bigskip
\begin{itemize}[<+->]
    \item Which of the economic-structure variables is associated with poverty rate?
    \item What are the sign and magnitude of that association?
    \item Is poverty rate associated with the same economic-structure variables across the entire region?
    \item Are the sign and magnitude of the associations constant across the region?
\end{itemize}

\note{
We're going to aim at answering these questions with the work I present today.

There are several other methods to answer at least some of these questions, which we'll cover next.
}
\end{frame}



\section{Introduction}



\begin{frame}{Introduction}
\subt{A review of existing methods}

\bigskip
\begin{itemize}[<+->]
    \item Spatial regression
    \item Varying coefficient regression
    \begin{itemize}
        \item Splines
        \item Kernels
        \item Wavelets
    \end{itemize}
    \item Model selection via regularization
\end{itemize}

\note{
Behind the methodology that I'm discussing is a wide range of literature.
}
\end{frame}






\begin{frame}{Introduction}
\subt{Some definitions}

\bigskip
\begin{itemize}
    \item Univariate spatial response process $\left\{ Y(\bm{s}) : \bm{s} \in \mathcal{D} \right\}$
    \item Multivariate spatial covariate process $\left\{ \bm{X}(\bm{s}) : \bm{s} \in \mathcal{D} \right\}$
    \item $n = $ number of observations
    \item $p = $ number of covariates
    \item Location (2-dimensional) $\bm{s}$
    \item Spatial domain $\mathcal{D}$
\end{itemize}


\note{
We'll use these variables throughout. 
}
\end{frame}





\begin{frame}{Introduction}
\subt{Further definitions}

\bigskip
\begin{itemize}
    \item Geostatistical data:
    \begin{itemize}
        \item Observations are made at sampling locations $\bm{s}_i$ for $i = 1, \dots, n$
        \item E.g. elevation, temperature
    \end{itemize}
    \item Areal data:
    \begin{itemize}
        \item Domain is partitioned into $n$ regions $\left\{ D_1, \dots, D_n \right\}$
        \item The regions do not overlap, and they divide the domain completely: $\mathcal{D} = \bigcup_{i=1}^n D_i$
        \item Sampling locations $\bm{s}_i$ for $i = 1, \dots, n$ are the centroids of the regions
        \item E.g. poverty rate, population, spatial mean temperature
    \end{itemize}
\end{itemize}


\note{
The method I'm describing applies to geostatistical data, or to areal data when the observations are assumed to be located at the centroid.

The poverty data example is areal data, the simulation study is based on simulated geostatistical data.
}
\end{frame}








\begin{frame}{Introduction}
\subt{Existing approaches: spatial regression}

\bigskip
\begin{itemize}
    \item The typical spatial regression \citep{Cressie:1993}
\end{itemize}
\begin{align}
    Y(\bm{s}) &= \bm{X}(\bm{s})'\bm{\beta} + W(\bm{s}) + \varepsilon(\bm{s}) \notag \\
    \text{cov}(W(\bm{s}), W(\bm{t})) &= \Gamma\left(\delta(\bm{s}, \bm{t})\right) \notag \\
    \delta(\bm{s}, \bm{t}) &= \sqrt{\|\bm{t} - \bm{s}\|_2} \notag \\
    \text{E.g. } \Gamma(\delta(\bm{s}, \bm{t})) &= \exp \{ -\phi^{-1} \delta(\bm{s}, \bm{t}) \}    
\end{align}
\pause
\begin{itemize}[<+->]
    \item $W(\bm{s})$ is a spatial random effect that accounts for autocorrelation in the response variable
    \item The coefficients $\bm{\beta}$ are constant
    \item Relies on \emph{a priori} global variable selection
\end{itemize}

\note{
This is the form of the usual spatial regression from e.g. Cressie (1993).

The spatial random effect W captures autocorrelation of the response, while the white noise is iid error

The Gamma function is a Matern-class covariance function, such as the exponential covariance function (listed here)
}
\end{frame}





\begin{frame}{Introduction}
\subt{Existing approaches: varying coefficients regression}

\[
    Y(\bm{s}) = \bm{X}(\bm{s})' \textcolor{hilight}{\bm{\beta}(\bm{s})} + \varepsilon(\bm{s})
\]

\pause
\begin{itemize}[<+->]
    \item Assume an effect modifying variable $S$
    \item Coefficients are functions of $S$
    \item Generally assume that the coefficient functions are smooth
    \item This is a varying coefficient regression (VCR) model \citep{Hastie:1993a}
    \item If $\bm{s}$ is a spatial location then we have a spatially varying coefficient regression (SVCR) model
\end{itemize}

\note{note}
\end{frame}






\begin{frame}{Introduction}
\subt{Existing approaches: spatially varying coefficient process}

\bigskip
\begin{itemize}
    \item Making model more flexible: coefficients in a spatial regression model can be allowed to vary \citep{Gelfand:2003}
\end{itemize}

\[
    Y(\bm{s}) = \bm{X}(\bm{s})' \textcolor{hilight}{\bm{\beta}(\bm{s})} + \varepsilon(\bm{s})
\]

\pause
\begin{itemize}[<+->]
    \item The spatial random effect has been incorporated into the spatially varying intercept
    \item $\{ \beta_1(\bm{s}) : \bm{s} \in \mathcal{D}\}, \dots, \{\beta_p(\bm{s}) : \bm{s} \in \mathcal{D}\}$ are stationary spatial processes with Mat\`{e}rn covariance functions
    \item Still relies on \emph{a priori} global variable selection
\end{itemize}

\note{note}
\end{frame}






\begin{frame}{Introduction}
\subt{Existing approaches: spline-based VCR and SVCR models}

\bigskip
\begin{itemize}
    \item Splines are a way to parameterize smooth functions
    \item Splines can be incorporated into a generalized additive model (GAM):
    \begin{itemize}
        \item $E\{Y(t)\} = f\{X_1(t)\} + \cdots + f\{X_p(t)\}$
    \end{itemize}
    \item It is possible to parameterize a VCR model with splines for the coefficient functions:
    \begin{itemize}
        \item $E\{Y(t)\} = \beta_1(t) X_1(t) + \cdots + \beta_p(t) X_p(t)$
    \end{itemize}
\end{itemize}

\note{note}
\end{frame}




\begin{frame}{Introduction}
\subt{Existing approaches: Global selection in spline-based VCR models}

\bigskip
Regularization methods for global variable selection in VCR models:
\begin{itemize}
    \item The integral of a function squared (e.g. $\int \{f(t)\}^2 dt$) is zero if and only if the function is zero everywhere.
    \item Use regularization (maximize the likelihood plus a penalty) to encourage coefficient functions to be zero
    \item SCAD penalty \citep{Fan:2001} on the integral of the square of the coefficient function \citep{Wang:2008a}
    \item Non-negative garrote penalty \citep{Breiman:1995} on the integral of the square of the coefficient function \citep{Antoniadis:2012b}
\end{itemize}

\note{These selection methods are all global - that is, they select variables for the entire domain simultaneously}
\end{frame}




\begin{frame}{Introduction}
\subt{Existing approaches: wavelet methods for VCR models}

\bigskip
Wavelet methods involve decomposing a function into local frequency components. Wavelet methods for VCR models include using Bayesian variable selection or the Lasso to estimate which local frequency components have nonzero coefficients \citep{Shang-2011,Zhang-2011}.

\vspace{10mm}

These methods achieve sparsity in the local frequency components but not in the local covariates, and so are not suitable for local variable selection.

\note{note}
\end{frame}




\section{Geographically weighted regression}



\begin{frame}{Geographically weighted regression}
\subt{Existing approaches: geographically weighted regression}

\bigskip
When the effect modifying variable $\bm{s}$ refers to spatial location, the method of local regression is called geographically weighted regression (GWR) \citep{Brundson:1998a, Fotheringham:2002}

\note{
Maximizing the local likelihood for a model of Gaussian data with iid errors can be done by weighted least squares.
}
\end{frame}



\begin{frame}{Geographically weighted regression}
\subt{Existing approaches: Local regression}

\bigskip
Local regression uses a kernel function at each sampling location to weight observations based on their distance from the sampling location. An example is the bisquare kernel:
\begin{align}
	w_{ii'} = \begin{cases} \left[1-\left(\phi^{-1}\delta_{ii'}\right)^2\right]^2 &\mbox{ if } \delta_{ii'} < \phi, \\ 0 &\mbox{ if } \delta_{ii'} \geq \phi. \end{cases}
\end{align}
Where $\phi$ is a bandwidth parameter.

Given the weights, a local model is fit at each sampling location using the local likelihood \citep{Loader:1999}

\note{note}
\end{frame}






\begin{frame}{Geographically weighted regression}
\subt{Existing approaches: Local likelihood}

\bigskip
Calibrate the model by doing the following at each sampling location:
\begin{itemize}
    \item Weight each observation's likelihood
    \item Weights are given by the kernel
\end{itemize}

\begin{align}
    L &= \prod_{i'=1}^n \left(L_{i'}\right)^{w_{ii'}} \notag \\
    \ell &= \sum_{i'=1}^n w_{ii'} \left\{\log \sigma_i^{2} + \sigma_i^{-2}\left(y_{i'} - \bm{x}_{i'}'\bm{\beta}_i\right)^2 \right\} \notag
\end{align}
Where $\bm{\beta}_i = \bm{\beta}(\bm{s}_i)$.

\note{
Maximizing the local likelihood for a model of Gaussian data with iid errors can be done by weighted least squares.
}
\end{frame}





\begin{frame}{Geographically weighted regression}
\subt{Existing approaches: bandwidth estimation for GWR}

\bigskip
\begin{itemize}
    \item Smaller bandwidth: less bias, more flexible coefficient surface
    \item Large bandwidth: less variance, less flexible coefficient surface
    \item Estimate the degrees of freedom used in estmating the coefficient surface \citep{Hurvich:1998}:
    \begin{itemize}
        \item $\hat{\bm{y}} = H\bm{y}$
        \item $\nu = $ tr($H$)
    \end{itemize}
    \item Then the corrected AIC for bandwidth selection is:
    \item $\text{AIC}_{\tt c} = 2 n \log{\sigma} + n \left\{\frac{n + \nu}{n - 2 - \nu}\right\}$
\end{itemize}

\note{
Maximizing the local likelihood for a model of Gaussian data with iid errors can be done by weighted least squares.
}
\end{frame}





\begin{frame}{Geographically weighted regression}
\subt{Existing approaches: geographically weighted Lasso}

\bigskip
Within a GWR model, using the Lasso \citep{Tibshirani:1996} for local variable selection is called the geographically weighted Lasso (GWL) \citep{Wheeler:2009}.

\begin{itemize}
    \item The GWL requires estimating a Lasso tuning parameter for each local model
    \item \cite{Wheeler:2009} estimates the local Lasso tuning parameter at location $\bm{s}_i$ by minimizing a jacknife criterion: $|y_i - \hat{y}_i|$
    \item The jacknife criterion can only be calculated where data are observed, making it impossible to use the GWL to impute missing data or to estimate the value of the coefficient surface at new locations
    \item Also, the Lasso is known to be biased in variable selection and suboptimal for coefficient estimation
\end{itemize}

\note{GWL does local variable selection}
\end{frame}



\section{Local variable selection and parameter estimation}



\begin{frame}{Local variable selection and  parameter estimation}
\subt{Geographically weighted elastic net (GWEN)}

\bigskip
\begin{itemize}
    \item Local variable selection in a GWR model using the adaptive elastic net (AEN) \citep{Zou:2009}
    \item Under suitable conditions, the AEN has an oracle property for selection
\end{itemize}

\begin{align}
		\mathcal{S}(\bm{\beta}_i) &= -2 \ell_i(\bm{\beta}_i) + \mathcal{J}_2(\bm{\beta}_i) \notag \\
		&= \sum_{i'=1}^n w_{ii'}  \left\{ \log{\sigma^2_i}  + \left(\sigma^2_i\right)^{-1}  \left(y_{i'} - \bm{x}'_{i'} \bm{\beta}_i \right)^2 \right\}   \notag \\
		&+ \alpha_i \lambda_i \sum_{j=1}^p |\beta_{ij}| / \gamma_{ij} \notag \\
		&+ (1-\alpha_i) \lambda^*_i \sum_{j=1}^p  \left( \beta_{ij} / \gamma_{ij} \right)^2 \notag
	\end{align}

\note{
The adaptive weights $\bm{\gamma}_i = \left(\gamma_{i1}, \dots, \gamma_{ip}\right)'$ are defined in the same way as for the AL, and the elastic net parameter $\alpha_i \in [0,1]$ controls the balance between $\ell_1$ penalty $\lambda^*_i\sum \limits_{j=1}^p |\beta_{ij}| / \gamma_{ij}$ and $\ell_2$ penalty $\lambda^*_i \sum_{j=1}^p  \left( \beta_{ij} / \gamma_{ij} \right)^2$.
}
\end{frame}



\begin{frame}{Local variable selection and  parameter estimation}
\subt{Geographically weighted elastic net (GWEN)}

\bigskip
where $\sum_{i'=1}^n w_{ii'} \left(y_{i'} - \bm{x}'_{i'} \bm{\beta}_i \right)^2$ is the weighted sum of squares minimized by traditional GWR, and $\mathcal{J}_2(\bm{\beta}_i) = \alpha_i \lambda^*_i \sum_{j=1}^p |\beta_{ij}| / \gamma_{ij} + (1-\alpha_i) \lambda^*_i \sum_{j=1}^p  \left( \beta_{ij} / \gamma_{ij} \right)^2$ is the AEN penalty.

\note{
The adaptive weights $\bm{\gamma}_i = \left(\gamma_{i1}, \dots, \gamma_{ip}\right)'$ are defined in the same way as for the AL, and the elastic net parameter $\alpha_i \in [0,1]$ controls the balance between $\ell_1$ penalty $\lambda^*_i\sum \limits_{j=1}^p |\beta_{ij}| / \gamma_{ij}$ and $\ell_2$ penalty $\lambda^*_i \sum_{j=1}^p  \left( \beta_{ij} / \gamma_{ij} \right)^2$.
}
\end{frame}





\begin{frame}{Local variable selection and  parameter estimation}
\subt{Geographically weighted elastic net (GWEN)}

\bigskip
It is necessary to estimate an AEN tuning parameter for each local model. Using the local BIC allows fitting a local model at any location within the domain
	\begin{align}\label{eq:BIC_loc}
		\mbox{BIC}_{\text{loc}, i} &= -2 \sum_{i'=1}^n \ell_{ii'}  + \log \left( \sum_{i'=1}^n w_{ii'} \right) \mbox{df}_i\notag\\
		&= -2 \sum_{i'=1}^n \log \left[ \left(2 \pi \hat{\sigma}_i^2\right)^{-1/2} \exp \left\{-\frac{1}{2} \hat{\sigma}_i^{-2} \left(y_{i'} - \bm{x}_{i'}' \hat{\bm{\beta}}_{i'} \right)^2\right\} \right]^{w_{ii'}} \notag \\
		&+ \log \left( \sum_{i'=1}^n w_{ii'} \right) \mbox{df}_i\notag\\
	\end{align}


\note{
The adaptive weights $\bm{\gamma}_i = \left(\gamma_{i1}, \dots, \gamma_{ip}\right)'$ are defined in the same way as for the AL, and the elastic net parameter $\alpha_i \in [0,1]$ controls the balance between $\ell_1$ penalty $\lambda^*_i\sum \limits_{j=1}^p |\beta_{ij}| / \gamma_{ij}$ and $\ell_2$ penalty $\lambda^*_i \sum_{j=1}^p  \left( \beta_{ij} / \gamma_{ij} \right)^2$.
}
\end{frame}







\begin{frame}{Local variable selection and  parameter estimation}
\subt{Geographically weighted elastic net (GWEN)}

\bigskip
	\begin{align}
		&= \sum_{i'=1}^n w_{ii'} \left\{ \log \left(2 \pi \right) + \log \hat{\sigma}_i^2 + \hat{\sigma}_i^{-2} \left(y_{i'} - \bm{x}_{i'}' \hat{\bm{\beta}}_{i'} \right)^2 \right\} \notag \\
		&+ \log \left( \sum_{i'=1}^n w_{ii'} \right) \mbox{df}_i \notag
	\end{align}


\note{We treat the sum of the weights around the sampling location as the number of observations for the local BIC.}
\end{frame}




\section{Simulation study}




\begin{frame}{Simulation study}
\subt{Simulating covariates}

\bigskip
Five covariates $\tilde{X}_1, \dots, \tilde{X}_5$ were simulated by Gaussian random fields on the domain $[0,1] \times [0,1]$ on a $30 \times 30$ grid of sampling locations:
\begin{align}
    \tilde{X}_j &\sim N(0, \Sigma) \text{ for } j = 1, \dots, 5 \notag \\
    \left\{\Sigma\right\}_{i,i'} &= \exp \{ -\tau^{-1} \delta_{ii'}\} \text{ for } i,i' = 1, \dots, n \notag
\end{align}

Where the covariates were simulated with colinearity, the colinearity was induced by multiplying the design matrix by the square root of the colinearity matrix:
\begin{align}
    \text{diag}(\Omega_{5\times5}) = 1 \notag \\
    \text{off-diag}(\Omega_{5\times5}) = \rho \notag \\
    X = \tilde{X} R
\end{align}
Where $\Omega_{5\times5} = R'R$ is the Cholesky decomposition.

\note{note}
\end{frame}


\begin{frame}{Simulation study}
\subt{Simulating the response}

\bigskip
\begin{itemize}
    \item $Y(\bm{s}) = X(\bm{s})'\bm{\beta}(\bm{s}) = \sum_{j=1}^5 \beta_j(\bm{s}) X_j(\bm{s}) + \varepsilon(\bm{s})$
    \item $\bm{\varepsilon} \sim \; iid \;\; N(0,\sigma^2) $ 
    \item $\beta_1(\bm{s})$, the coefficient function for $X_1$, is nonzero in part of the domain.
    \item Coefficients for $X_2, \dots, X_5$ are zero everywhere
\end{itemize}

\note{note}
\end{frame}



\begin{frame}{Simulation study}
\subt{Coefficient functions}

\bigskip
Call these functions step, gradient, and parabola:

\begin{figure}
    \begin{center}
    \ig[width=0.35\textwidth]{../../figures/practice-talk/step.jpg}
    \ig[width=0.35\textwidth]{../../figures/practice-talk/gradient.jpg}
    \ig[width=0.35\textwidth]{../../figures/practice-talk/parabola.jpg}
    \end{center}
\end{figure}

\note{note}
\end{frame}




\begin{frame}{Simulation study}
\subt{Simulation settings}
    % latex table generated in R 2.15.1 by xtable 1.7-0 package
    % Fri Jan 18 10:19:47 2013
    \begin{table}[h!]
        \begin{center}
        \begin{tabular}{cccc}
            \hline
            Setting & function & $\rho$ & $\sigma^2$ \\ 
            \hline
            1 & step & 0 & 0.25 \\ 
            2 & step & 0 & 1 \\ 
            3 & step & 0.5 & 0.25 \\ 
            4 & step & 0.5 & 1 \\ 
            \hline
            5 & gradient & 0 & 0.25 \\ 
            6 & gradient & 0 & 1 \\ 
            7 & gradient & 0.5 & 0.25 \\ 
            8 & gradient & 0.5 & 1 \\ 
            \hline
            9 & parabola & 0 & 0.25 \\ 
            10 & parabola & 0 & 1 \\ 
            11 & parabola & 0.5 & 0.25 \\ 
            12 & parabola & 0.5 & 1 
        \end{tabular}
        \end{center}
        \caption{Simulation parameters for each setting.\label{table:simulation_settings}}
    \end{table}
\note{note}
\end{frame}


\begin{frame}{Simulation results}
\subt{Selection}

\begin{figure}
    \begin{center}
    \ig[width=0.65\textwidth]{../../figures/simulation/summary-locations}
    \end{center}
\end{figure}

\note{note}
\end{frame}


\begin{frame}{Simulation results}
\subt{Selection}
\input{../../output/prelim-talk/selection}
\note{note}
\end{frame}




\begin{frame}{Simulation results}
\subt{MSE of $\beta_1(\bm{s})$ - step coefficient surface}
\input{../../output/prelim-talk/msex-step}
\note{note}
\end{frame}



\begin{frame}{Simulation results}
\subt{MSE of $\beta_1(\bm{s})$ - gradient coefficient surface}
\input{../../output/prelim-talk/msex-gradient}
\note{note}
\end{frame}


\begin{frame}{Simulation results}
\subt{MSE of $\beta_1(\bm{s})$ - parabola coefficient surface}
\input{../../output/prelim-talk/msex-parabola}
\note{note}
\end{frame}





\begin{frame}{Simulation results}
\subt{Variance of $\beta_1(\bm{s})$ - step coefficient surface}
\input{../../output/prelim-talk/varx-step}
\note{note}
\end{frame}


\begin{frame}{Simulation results}
\subt{Variance of $\beta_1(\bm{s})$ - gradient coefficient surface}
\input{../../output/prelim-talk/varx-gradient}
\note{note}
\end{frame}


\begin{frame}{Simulation results}
\subt{Variance of $\beta_1(\bm{s})$ - parabola coefficient surface}
\input{../../output/prelim-talk/varx-parabola}
\note{note}
\end{frame}




\begin{frame}{Simulation results}
\subt{Bias of $\beta_1(\bm{s})$ - step coefficient surface}
\input{../../output/prelim-talk/bx-step}
\note{note}
\end{frame}


\begin{frame}{Simulation results}
\subt{Bias of $\beta_1(\bm{s})$ - gradient coefficient surface}
\input{../../output/prelim-talk/bx-gradient}
\note{note}
\end{frame}


\begin{frame}{Simulation results}
\subt{Bias of $\beta_1(\bm{s})$ - parabola coefficient surface}
\input{../../output/prelim-talk/bx-parabola}
\note{note}
\end{frame}



\section{Data example: poverty rate in the upper midwest}



\begin{frame}{Data example: poverty rate in the upper midwest}
\subt{Revisiting the introductory example}

\bigskip
\begin{center}
  \ig[width=\textwidth]{../../figures/practice-talk/poverty-covariates}
\end{center}

\note{
This is the county-level poverty rate from 1970, as well as the proportion of people who worked in manufacturing, agriculture, and services.

How is this data to be analyzed?
}
\end{frame}






\begin{frame}{Data example: poverty rate in the upper midwest}
\subt{Results from traditional GWR}

\begin{center}
  \ig[width=\textwidth]{../../figures/practice-talk/1970-GWR-coefficients}
\end{center}

\note{note}
\end{frame}






\begin{frame}{Data example: poverty rate in the upper midwest}
\subt{Results from GWEN}

\begin{center}
  \ig[width=\textwidth]{../../figures/practice-talk/1970-GWEN-coefficients}
\end{center}

\note{note}
\end{frame}



\section{Future work}


\begin{frame}{Future work}
\begin{itemize}
    \item Apply the GWEN to data with non-Gaussian response variable
    \item Incorporate spatial autocorrelation in the model (simulated errors were iid)
\end{itemize}
\note{note}
\end{frame}



\begin{frame}{Acknowledgements}

\end{frame}




\end{document}
